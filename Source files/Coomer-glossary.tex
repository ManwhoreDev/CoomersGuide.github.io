% This file contains glossary definitons.
% Once any of these gets invoked in inside document, the will be added to Glossary chapter.
% Add new entries according to an example.
% First value in brackets is used to refer to glossary term via macro so keeping it short is preferrable.
% \gls{gpt3} will print a colored link to glossary page containing ``GPT-3'' entry. Th
% \gls{ex} will print ``Example'' in the document, the 'name' value in glossary macro.
% \gls can be used recursively, to refer to another Glossary entry,
% It will be userful once they are located on different pages.
% Don't forget to add a dot after using \gls at the end.

\newglossaryentry{ex}{%
	name={Example},
	description={An example of a Glossary entry.}
}

\newglossaryentry{ex1}{%
	name={A boring Example},
	description={A boring kind of a Glossary entry.}
}

\newglossaryentry{ex2}{%
	name={A Cool example},
	description={A cool kind of a Glossary entry. Don't mistake it for \gls{ex1}.} % <--- Dot between two brackets }.} !!!!
}

